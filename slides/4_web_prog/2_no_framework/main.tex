\documentclass[c,unicode,russian]{beamer}
\usepackage{hyperref}
\usepackage{alltt}
\usepackage{verbatim}
\usepackage{fancyvrb}

\usepackage{fontspec}
\setsansfont{Ubuntu}
\setmonofont{Ubuntu Mono}
\usepackage{polyglossia}
\setdefaultlanguage{russian}

\useinnertheme{metropolis}
\useoutertheme{metropolis}
\usecolortheme{metropolis}

\usepackage{listings,xcolor}
\usepackage{../../listings-golang/listings-golang}   % Golang
\lstset{%
    keywordstyle=\color{blue},
    commentstyle=\color[rgb]{0.13,0.54,0.13},
    backgroundcolor=\color{yellow!10},
    basicstyle=\small\tt,
    stringstyle=\color{red}\ttfamily,
    showstringspaces=false,
    belowcaptionskip=-1pt,
    xleftmargin=-15pt,
    framexleftmargin=-15pt,
    framexrightmargin=5pt,
    framextopmargin=5pt,
    framexbottommargin=5pt,
    framesep=0pt,
    rulesep=0pt
}
\lstdefinestyle{cpp}{%
    language=C++,
    morecomment=[l][\color{magenta}]{\#}
}
\lstdefinestyle{python}{%
    language=Python
}
\lstdefinestyle{php}{%
    language=php
}
\lstdefinestyle{html}{%
    language=html
}
\lstdefinestyle{go}{% add your own preferences
    language=Golang
}

\usepackage{caption}
\renewcommand{\lstlistingname}{Код} % Listing -> Algorithm
\DeclareCaptionFont{white}{\color{white}}
\DeclareCaptionFormat{listing}{\colorbox{gray}{\parbox{\textwidth}{#1#2#3}}}
\captionsetup[lstlisting]{format=listing,labelfont=white,textfont=white}

% logo of my university
\titlegraphic{\vspace{-35pt}\hspace{-1cm}\includegraphics[width=\paperwidth]{../../_static/logo.png}}

\date{}
\author{Основы Веб-программирования}
\institute{Кафедра Интеллектуальных Информационных Технологий, ИнФО, УрФУ}

\usepackage{array}      % Table

\title{Веб без фреймворков}

\begin{document}

% Slide #1
\frame{\titlepage}

% Slide #2
\begin{frame}{Ресурсы}
  \url{http://lectures.uralbash.ru/6.www.sync/2.codding/index.html}
\end{frame}

% Slide #3
\begin{frame}{WSGI - это\ldots?}

    Для разработки сайтов или Web-приложений на языке Python был утверждён
    стандарт взаимодействия между Python-приложениями и сервером (например
    Apache), названный WSGI (“Web Server Gateway Interface”).

    \textbf{Python}\newline
    \textbf{pep-333}\newline
    \textbf{pep-3333}\newline

\end{frame}

% Slide #4
\begin{frame}{Общие принципы}

  \begin{itemize}
    \item Веб-сервер
    \item Разделение кода: \textbf{MVC}, \textbf{MTV}, \textbf{RV}
    \item Маршрутизация URL
    \item Шаблоны
    \item Пагинация
    \item Request/Response
    \item Статика
    \item Формы
  \end{itemize}

\end{frame}

% Slide #5
\begin{frame}{Веб-сервер}

  Задача Веб сервера - запускать Веб приложения.\newline\newline
    Популярные WSGI Веб сервера:\newline

  \begin{itemize}
    \item wsgiref
    \item Paste
    \item Waitress
    \item Gunicorn
  \end{itemize}


\end{frame}

% Slide #6
\begin{frame}[fragile]{wsgiref}

    \begin{lstlisting}[style=python]
    from wsgiref.simple_server import make_server

    def hello_world_app(environ, start_response):
        status = '200 OK'  # HTTP Status
        headers = [
          ('Content-type', 'text/plain; charset=utf-8')
        ]  # HTTP Headers
        start_response(status, headers)

        # The returned object is going to be printed
        return [b"Hello World"]

    with make_server('', 8000, hello_world_app) as httpd:
        print("Serving on port 8000...")

        # Serve until process is killed
        httpd.serve_forever()
    \end{lstlisting}

\end{frame}

% Slide #7
\begin{frame}{Разделение кода: \textbf{MVC}, \textbf{MTV}, \textbf{RV}}

  \textbf{MVC} (Model-View-Controller)

  123

\end{frame}


% Slide #8
\begin{frame}{Разделение кода: \textbf{MVC}, \textbf{MTV}, \textbf{RV}}

  M -> M
  V -> T
  C -> V

  Tada! Django MTV

\end{frame}

% Slide #9
\begin{frame}{Разделение кода: \textbf{MVC}, \textbf{MTV}, \textbf{RV}}

  classic MVC:

  Pylons, Rails

\end{frame}

% Slide #11
\begin{frame}{Разделение кода: \textbf{MVC}, \textbf{MTV}, \textbf{RV}}

  Разработка без фреймворков дает вам возможность придерживаться любой
  архитектуры приложения, независимо от паттерна проектирования.

\end{frame}

% Slide #12
\begin{frame}{Разделение кода: \textbf{MVC}, \textbf{MTV}, \textbf{RV}}

  Это дает неоспоримую гибкость таким приложениям.

\end{frame}


% Slide #10
\begin{frame}{Разделение кода: \textbf{MVC}, \textbf{MTV}, \textbf{RV}}

  RV - дает ту же гибкость, накладывая минимальную архитектуры идеально
  вписывающуюся в ограничения Веб приложений.

\end{frame}


% Slide #10
\begin{frame}{Разделение кода: \textbf{MVC}, \textbf{MTV}, \textbf{RV}}

  RV - WTF?????

  Адреса - пути.
  Ресурсы привязаны к путям.
  View обрабатывают ресурсы.

  Pyramid only

\end{frame}



\end{document}
