\documentclass[c,unicode,russian]{beamer}
\usepackage{hyperref}
\usepackage{alltt}
\usepackage{verbatim}
\usepackage{fancyvrb}

\usepackage{fontspec}
\setsansfont{Ubuntu}
\setmonofont{Ubuntu Mono}
\usepackage{polyglossia}
\setdefaultlanguage{russian}

\useinnertheme{metropolis}
\useoutertheme{metropolis}
\usecolortheme{metropolis}

\usepackage{listings,xcolor}
\usepackage{../../listings-golang/listings-golang}   % Golang
\lstset{%
    keywordstyle=\color{blue},
    commentstyle=\color[rgb]{0.13,0.54,0.13},
    backgroundcolor=\color{yellow!10},
    basicstyle=\small\tt,
    stringstyle=\color{red}\ttfamily,
    showstringspaces=false,
    belowcaptionskip=-1pt,
    xleftmargin=-15pt,
    framexleftmargin=-15pt,
    framexrightmargin=5pt,
    framextopmargin=5pt,
    framexbottommargin=5pt,
    framesep=0pt,
    rulesep=0pt
}
\lstdefinestyle{cpp}{%
    language=C++,
    morecomment=[l][\color{magenta}]{\#}
}
\lstdefinestyle{python}{%
    language=Python
}
\lstdefinestyle{php}{%
    language=php
}
\lstdefinestyle{html}{%
    language=html
}
\lstdefinestyle{go}{% add your own preferences
    language=Golang
}

\usepackage{caption}
\renewcommand{\lstlistingname}{Код} % Listing -> Algorithm
\DeclareCaptionFont{white}{\color{white}}
\DeclareCaptionFormat{listing}{\colorbox{gray}{\parbox{\textwidth}{#1#2#3}}}
\captionsetup[lstlisting]{format=listing,labelfont=white,textfont=white}

% logo of my university
\titlegraphic{\vspace{-35pt}\hspace{-1cm}\includegraphics[width=\paperwidth]{../../_static/logo.png}}

\date{}
\author{Основы Веб-программирования}
\institute{Кафедра Интеллектуальных Информационных Технологий, ИнФО, УрФУ}


\title{Базы данных}

\begin{document}

\frame{\titlepage}

\begin{frame}{Ресурсы}
  \url{http://lectures.uralbash.ru/6.www.sync/2.codding/9.databases/index.html}
\end{frame}

\begin{frame}{DB-API 2.0}

  \textbf{DB-API} - общий правила работы с БД.

  \textbf{pep-249}\newline

\end{frame}

\begin{frame}{DB-API 2.0}

  \begin{itemize}
    \item Connection
    \item Cursor
    \item Типы данных
    \item Исключения
  \end{itemize}

\end{frame}

\begin{frame}{DB-API 2.0 - Connection}

  \textbf{Connection} - устанавливает соединение с БД.

\end{frame}

\begin{frame}{DB-API 2.0 - Connection. Операции}

  \begin{itemize}

    \item \textbf{close()}- Закрывает соединение с базой данных.
    \item \textbf{commit()} - Завершает транзакцию.
    \item \textbf{rollback()} - Откатывает начатую транзакцию
    \item \textbf{cursor()} - Возвращает объект-курсор, использующий данное
      соединение.  Если база данных не поддерживает курсоры, модуль сопряжения
      должен их имитировать.

  \end{itemize}

\end{frame}

\begin{frame}[fragile]{DB-API 2.0 - Connection. Пример}

    \begin{lstlisting}[style=python]
    import sqlite3
    conn = sqlite3.connect('example.sqlite')
    \end{lstlisting}

    \begin{lstlisting}[style=python]
    import sqlite3
    conn = sqlite3.connect(':memory:')
    \end{lstlisting}

\end{frame}

\begin{frame}{DB-API 2.0 - Cursor}

  \textbf{Курсор} — объект базы данных, который позволяет приложениям работать
  с записями «по одной», а не сразу с множеством, как это делается в обычных
  SQL командах

\end{frame}

\begin{frame}{DB-API 2.0 - Cursor. Операции}

  \begin{itemize}

    \item \textbf{execute()} - Закрывает соединение с базой данных.
    \item \textbf{executemany()} - Выполняет серию запросов или команд.
    \item \textbf{callproc()} - Откатывает начатую транзакцию
    \item \textbf{cursor()} - Вызывает хранимую процедуру.

  \end{itemize}

\end{frame}

\begin{frame}[fragile]{DB-API 2.0 - Cursor. Пример}

    \begin{lstlisting}[style=python]
    c = conn.cursor()

    # Создание таблицы
    c.execute('''CREATE TABLE stocks
    (date text, trans text,
    symbol text, qty real, price real)''')

    # Добавление записи
    c.execute(
      "INSERT INTO stocks VALUES"
      "('2006-01-05','BUY','RHAT',100,35.14)"
    )
    \end{lstlisting}

\end{frame}

\begin{frame}[fragile]{DB-API 2.0 - Cursor. Пример}

    \begin{lstlisting}[style=python]
    # Сохранение (commit) изменений
    conn.commit()

    # Закрытие соединения.
    # Если изменения не были сохранены (метод commit),
    # то данные пропадут.
    conn.close()
    \end{lstlisting}

\end{frame}

\begin{frame}[fragile]{DB-API 2.0 - Cursor. Пример}

    \begin{lstlisting}[style=python]
    # Запись сразу нескольких объектов за раз
    purchases = [
     ('2006-03-28', 'BUY', 'IBM', 1000, 45.00),
     ('2006-04-05', 'BUY', 'MSFT', 1000, 72.00),
     ('2006-04-06', 'SELL', 'IBM', 500, 53.00),
    ]
    c.executemany(
      'INSERT INTO stocks VALUES (?,?,?,?,?)',
      purchases
    )
    \end{lstlisting}

\end{frame}

\begin{frame}{DB-API 2.0 - Cursor. Результат}

  \begin{itemize}

    \item \textbf{fetchone()} - Возвращает следующую запись (в виде
      последовательности) из результата запроса.
    \item \textbf{fetchall()} - Возвращает все (или все оставшиеся) записи
      результата запроса.
    \item \textbf{fetchmany()} - Возвращает следующие несколько записей из
      результатов запроса.

  \end{itemize}

\end{frame}

\begin{frame}[fragile]{DB-API 2.0 - Cursor. Пример}

    \begin{lstlisting}[style=python]
    # Никогда так не делайте -- не безопасно!
    symbol = 'RHAT'
    c.execute(
    "SELECT * FROM stocks WHERE symbol = '%s'" % symbol
    )

    # Правильно
    t = ('RHAT',)
    c.execute(
      'SELECT * FROM stocks WHERE symbol=?',
      t
    )
    print(c.fetchone())
    \end{lstlisting}

\end{frame}

\begin{frame}{DB-API 2.0 - Типы}

  \begin{itemize}

    \item \textbf{Date}
    \item \textbf{Time}
    \item \textbf{Timestamp}
    \item \textbf{DateFromTicks}
    \item \textbf{TimeFromTicks}
    \item \textbf{Binary}

  \end{itemize}

\end{frame}

\begin{frame}[fragile]{DB-API 2.0 - Исключения}

    \begin{lstlisting}[style=python]
    StandardError
    ├──Warning
    └──Error
       ├──InterfaceError (a problem with the db api)
       └──DatabaseError (a problem with the database)
          ├──DataError (bad data, values out of range, etc.)
          ├──OperationalError (the db has an issue out of our control)
          ├──IntegrityError
          ├──InternalError
          ├──ProgrammingError (something wrong with the operation)
          └──NotSupportedError (the operation is not supported)
    \end{lstlisting}

\end{frame}

\begin{frame}{DB-API 2.0 - Кто использует}

  \begin{itemize}

    \item SQLite
    \item PostgreSQL (psycopg2, txpostgres, …)
    \item MySQL (mysql-python, PyMySQL, …)
    \item MS SQL Server (adodbapi, pymssql, mxODBC, pyodbc, …)
    \item Oracle (cx\_Oracle, mxODBC, pyodbc, …)
    \item и другие http://wiki.python.org/moin/DatabaseInterfaces

  \end{itemize}

\end{frame}

\end{document}
